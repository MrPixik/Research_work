\chapter*{Аннотация}
\label{ch:intro}
В настоящей работе представлены результаты научного исследования, направленного на изучение зон активности головного мозга для различных диапазонов частот с посредством анализа данных электроэнцефалограмм. Основной целью исследования являлось выявление областей головного мозга с минимальной активностью.

Для выполнения поставленных задач применялась библиотека для Python MNE, посредством которой были реализованы программы, обеспечивающие считывание, предобработку и визуализацию данных электроэнцефалографии. В рамках анализа были проведены процедуры устранения низкочастотных дрейфов, а также сетевых и других узкополосных помех. Дополнительно использовалась библиотека для Pyhon OpenCV (CV2) совместно с разработанным инструментарием для обработки изображений и выделения контуров.

В результате исследования установлено, что наименее активной зоной является центральная зона головного мозга, отвечающая за моторную активность.
\vspace*{-\baselineskip}
