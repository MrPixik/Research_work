\chapter*{Введение}
\addcontentsline{toc}{chapter}{Введение}
\label{ch:intro}

 Электроэнцефалограмма (ЭЭГ) — метод регистрации электрической активности головного мозга, который применяется для диагностики неврологических расстройств, мониторинга функционального состояния мозга и научных исследований в области нейрофизиологии. В медицинской практике она используется для диагностики эпилепсии, расстройств сна, оценивания состояния мозга после травм, инсультов и других паталогических состояний. В нейрофизиологии и когнитивных науках ЭЭГ применяется для изучения функций мозга, таких как внимание, восприятие и память.

ЭЭГ позволяет получить информацию о временной динамике и распределении электрической активности мозга с помощью электродов, размещенных на поверхности головы. Сигналы, записываемые в процессе ЭЭГ, представляют собой колебания потенциалов, возникающих в результате активности нейронных связей. Эти колебания могут быть классифицированы по частотным диапазонам, которые соответствуют различным состояниям мозга и когнитивным процессам. Основные частотные диапазоны, выделяемые в ЭЭГ-сигналах, включают:
\begin{enumerate}
    \item Дельта-ритм (Delta):
    Частотный диапазон от 0 до 4 Гц. Дельта-ритм связан с глубоким сном, состоянием покоя и восстановления. В норме он наблюдается у здоровых людей только в состоянии глубокого сна, но у пациентов с эпилепсией или органическими поражениями мозга может проявляться в состоянии бодрствования.
    \item Тета-ритм (Theta):
     Частотный диапазон от 4 до 8 Гц. Тета-ритм ассоциируется с состоянием расслабления, медитацией, сном и эмоциональными процессами. Он также связан с памятью и когнитивным функционированием, особенно в контексте релаксации и сна.
    \item Альфа-ритм (Alpha):
    Частотный диапазон от 8 до 12 Гц. Альфа-ритм является наиболее изученным и связан с состоянием спокойного бодрствования, расслабления и отсутствия внешних стимулов. Он наиболее выражен в затылочных областях и уменьшается при возникновении внимания или активации.
    \item Бета-ритм (Beta):
    Частотный диапазон от 12 до 30 Гц. Бета-ритм связан с активным состоянием бодрствования, когнитивной нагрузкой, вниманием и обработкой информации. Он отражает активацию корковых зон, участвующих в высших когнитивных функциях.
    \item Гамма-ритм (Gamma):
    Частотный диапазон от 30 до 45 Гц.. Гамма-ритм связан с высокоуровневыми когнитивными процессами, такими как восприятие, внимание, обработка сложной информации и интеграция сенсорных данных. В последние годы исследования указывают на важную роль гамма-активности в когнитивных и эмоциональных процессах.
\end{enumerate}
Однако, кроме полезных сигналов, в данные ЭЭГ попадают различные помехи, которые необходимо устранить для корректного анализа. Одним из наиболее часто встречающихся типов помех являются низкочастотные дрейфы, которые обычно имеют частоту ниже 1 Гц и могут быть вызваны неравномерным дыханием, изменениями кровотока или артериального давления, а также изменениями температуры тела. Эти помехи проявляются в виде медленных изменений базовой линии сигнала. Другой распространенный тип помех — мышечные артефакты, обусловленные активностью мышц головы, таких как жевательные мышцы или движения глаз, которые накладывают высокочастотные шумы на ЭЭГ-сигналы. Фильтрация сигналов осуществляется на основе преобразования Фурье, которое позволяет разделить сигнал на составляющие частоты. Удаляя нежелательные частотные компоненты и восстанавливая сигнал с оставшимися частотами, можно получить очищенные данные для дальнейшего анализа.

Топографическая карта головного мозга — это визуализация распределения электрической активности мозга на поверхности головы. Такие карты строятся на основе амплитуд сигналов ЭЭГ, зарегистрированных с разных электродов, и позволяют оценить пространственную организацию активности мозга. Топографические карты широко используются для выявления функциональных зон мозга, оценки межполушарной асимметрии и анализа локальных изменений активности, связанных с различными когнитивными процессами или патологическими состояниями.

%  В данном семестре был выполнен ряд задач:\newline
%1. Освоение библиотеки MNE для работы с ЭЭГ-данными и разработка программы их последующей %обработки.\newline
%2. Разработка алгоритма для сохранения топографических карт с заданным временным интервалом.\newline
%3. Разработка алгоритма для выявления неактивных участков мозга на основе полученных %топографических карт.\newline
%4. Разработка алгоритма создания видео с визуализацией неактивных участков.

\endinput